\documentclass[11pt]{article}

\usepackage{graphicx}
\usepackage{amsmath,amsfonts,amssymb}

\usepackage{hyperref}  % for urls and hyperlinks


\setlength{\textwidth}{6.2in}
\setlength{\oddsidemargin}{0.3in}
\setlength{\evensidemargin}{0in}
\setlength{\textheight}{8.9in}
\setlength{\voffset}{-1in}
\setlength{\headsep}{26pt}
\setlength{\parindent}{0pt}
\setlength{\parskip}{5pt}

% input some useful macros from RJLmacros.tex:
\input{../hw1/RJLmacros}

\begin{document}

\hfill\vbox{\hbox{AMath 574}\hbox{Homework 3}
\hbox{Due by 11:00pm on February 9, 2017}}

For submission instructions, see:

\url{http://faculty.washington.edu/rjl/classes/am574w2017/homework3.html}



%--------------------------------------------------------------------------

\vskip 1cm
\hrule
{\bf Problem \#1}

Consider the scalar conservation law with flux function $f(q) = q^3 - 4q^2 +
3q = q(q-1)(q-3)$.

(a) Show that the flux is convex as long as we restrict the data to fall
within $-\infty < q < 4/3$ or within $4/3 < q < +\infty$.

(b) Determine the exact solution to the Riemann problem with data $q_\ell =
3, q_r = 2$.  

(c) Determine the exact solution to the Riemann problem with data $q_\ell =
2, q_r = 3$.  In this case the solution is a rarefaction wave, so determine
the solution in the form of a similarity solution $q(x,t) = Q(x/t)$ and find
an exact expression for the function $Q(\xi)$. At the trailing edge of the
rarefaction wave there is a kink, a jump in the slope of the solution, from
0 to some non-zero value.  What value (as a function of time)?

(d) Re-do part (c) when $q_\ell = 4/3,~ q_r = 3$.  
Comment on why you expect the slope at the trailing edge of the rarefaction
wave to be infinite in this case.
Also plot the solution as a function of $x$ at time $t=1$.

(e) Consider this same conservation law with initial data
\[
q(x,0) = \begin{cases} 
        4 &\text{if~}x<0,\\ 
        3 &\text{if~}0<x<1,\\
        2 &\text{if~}x>1
         \end{cases}
\]
The solution consists of two shocks that merge at some time --- determine
the time when they merge, and the new shock speed.


% uncomment the next two lines if you want to insert solution...
%\vskip 1cm
%{\bf Solution:}

% insert your solution here!

%--------------------------------------------------------------------------
\vskip 1cm
\hrule
{\bf Problem \#2.}

Some sample code in \verb+$AM574/homeworks/hw3/burgers+ should help
get you started with this problem.  I will also provide a video.

Modify the code provided to solve the conservation law from Problem \#1
above, assuming that the initial data satisfies $q(x,0) \geq 4/3$ for all
$x$.

Test it out on $-2 \leq x \leq 2$  with initial data
\[
q(x,0) = \begin{cases} 
        4/3 &\text{if~}-2\leq x<0,\\ 
        3 &\text{if~}0\leq x\leq 2,
         \end{cases}
\]
and {\em periodic boundary conditions}.   
Solve the problem up to time $t=0.25$.

Modify the plotting specified in the provided {\tt setplot.py} file so that
you plot the true solution to this problem along with the approximate
solution.


\vskip 5pt
Experiment with this code and comment on what you observe.  You might want
to try:

\begin{itemize}
\item Changing the grid resolutions, e.g. using 100, 200, or more grid cells.

\item Use the first order accurate method (specify {\tt clawdata.order = 1})
in {\tt setrun.py} and compare to the high resolution 
method with {\tt clawdata.order = 2} and {\tt clawdata.limiter = ['mc']}.

\item Compare results with and without the entropy fix.

\end{itemize} 

What to turn in:
\begin{itemize}
\item Create a new directory \verb+cubic+ that has the necessary files, in
particular {\tt Makefile, setrun.py, setplot.py, rp1\_cubic.f90, qinit.f90,
setprob.f90}.  Note that {\tt rp1\_cubic.f90} will be a modified version of
{\tt rp1\_burgers.f90}.  
\item Turn in a tar file of this directory, set up for the case where the
high-resolution method is used with the entropy fix and 100 grid cells.
\item Provide plots of a few other key results to illustrate your
discussion.
\item One way to write up your observations would be to
modify the Jupyter notebook {\tt burgers.ipynb} to make a new notebook {\tt
cubic.ipynb} that contains some discussion and examples illustrating 
your results.
\end{itemize} 

% uncomment the next two lines if you want to insert solution...
%\vskip 1cm
%{\bf Solution:}

% insert your solution here!


%--------------------------------------------------------------------------
\newpage
\vskip 1cm
\hrule
{\bf Problem \#3.}

Suppose the solution to a Riemann problem for some system of conservation
laws $q_t +f(q)_x = 0$ 
consists of exactly 2 waves with wave speeds $s^1 < 0 < s^2$.  
Given Riemann data $q_\ell$ and $q_r$, let $q_m$ be the resulting state
between the two waves.
Then we can integrate over the space-time region shown below in order to
find a simple expression for $q_m$ in terms of $q_\ell,~ q_r, ~s^1,$ and
$s^2$ (in a similar manner to how integrating 
over the region shown in Figure 11.7 gives the Rankine-Hugoniot condition).

\hfil\includegraphics[width=4.0in]{qbox.png}\hfil


(a) Use this approach to compute the formula for $q_m$.

(b) Define two waves by ${\cal W}^1 = q_m-q_\ell$ and ${\cal W}^2 = q_r -
q_m$.  Show that
\[
s^1{\cal W}^1 + s^2 {\cal W}^2 = f(q_r) - f(q_\ell).
\]

(c) Apply the formula from (a) to the case of constant coefficient
linear acoustics with $s^1 = -c$ and $s^2 =
+c$ and show that the resulting $q_m$ agrees with what was found in (3.32)
in the book based on the eigenvectors of the coefficient matrix.

{\em Note:} For any system of $m$ equations we could choose any values $s^1
< s^2$ and use the formula you found to define a state $q_m$ and hence
define two waves ${\cal W}^1 = q_m-q_\ell$ and ${\cal W}^2 = q_r - q_m$.
These waves could then be used to obtain a conservative method (which
follows from (b), as we will see later).  Limiters
and high-resolution correction terms can also be based on this waves.

This won't be the exact Riemann solution except for special cases like the
acoustics equation (or any linear system of two equations where $s^1$ and
$s^2$ are chosen to be the two eigenvalues).  But it defines an {\em
approximate Riemann solver} that is very cheap to compute and sometimes
works well enough.  This is called the {\em HLL solver} after the original
work on this idea by Harten, Lax, and van Leer.  This is discussed in
Section 15.3.7 along with some extensions. (You'll find the solution to part
(a) there too.)

% uncomment the next two lines if you want to insert solution...
%\vskip 1cm
%{\bf Solution:}

% insert your solution here!


\end{document}

