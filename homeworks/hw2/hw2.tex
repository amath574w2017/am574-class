\documentclass[11pt]{article}

\usepackage{graphicx}
\usepackage{amsmath,amsfonts,amssymb}

\usepackage{hyperref}  % for urls and hyperlinks


\setlength{\textwidth}{6.2in}
\setlength{\oddsidemargin}{0.3in}
\setlength{\evensidemargin}{0in}
\setlength{\textheight}{8.9in}
\setlength{\voffset}{-1in}
\setlength{\headsep}{26pt}
\setlength{\parindent}{0pt}
\setlength{\parskip}{5pt}

% input some useful macros from RJLmacros.tex:
\input{../hw1/RJLmacros}

\begin{document}

\hfill\vbox{\hbox{AMath 574}\hbox{Homework 2}
\hbox{Due by 11:00pm on February 2, 2017}}

For submission instructions, see:

\url{http://faculty.washington.edu/rjl/classes/am574w2017/homework2.html}


%--------------------------------------------------------------------------
\vskip 1cm
\hrule
{\bf Problem \#6.1 in the book}


% uncomment the next two lines if you want to insert solution...
%\vskip 1cm
%{\bf Solution:}

% insert your solution here!

%--------------------------------------------------------------------------
\vskip 1cm
\hrule
{\bf Problem \#6.2 in the book}


% uncomment the next two lines if you want to insert solution...
%\vskip 1cm
%{\bf Solution:}

% insert your solution here!

%--------------------------------------------------------------------------

\vskip 1cm
\hrule
{\bf Problem \#6.5 in the book}


% uncomment the next two lines if you want to insert solution...
%\vskip 1cm
%{\bf Solution:}

% insert your solution here!


%--------------------------------------------------------------------------
\vskip 1cm
\hrule
{\bf Problem \#6.6 in the book}

% uncomment the next two lines if you want to insert solution...
%\vskip 1cm
%{\bf Solution:}

% insert your solution here!

%--------------------------------------------------------------------------
\vskip 1cm
\hrule
{\bf Problem \#6.7 in the book}

% uncomment the next two lines if you want to insert solution...
%\vskip 1cm
%{\bf Solution:}

% insert your solution here!

%--------------------------------------------------------------------------
\vskip 1cm
\hrule
{\bf Problem \#7.1 in the book}

% uncomment the next two lines if you want to insert solution...
%\vskip 1cm
%{\bf Solution:}

% insert your solution here!

%--------------------------------------------------------------------------
\vskip 1cm
\hrule
{\bf Problem \#7.2 in the book}

% uncomment the next two lines if you want to insert solution...
%\vskip 1cm
%{\bf Solution:}

% insert your solution here!

%--------------------------------------------------------------------------

\vskip 1cm
\hrule
{\bf Additional Problem:}

Show that the flux-limiter method (6.40),(6.41) can be written as a {\em
wave limiter} method as:

\[
Q_i^{n+1} = Q_i^n - \dtdx (\bar u^+ \W_{i-1/2}  + \bar u^- \W_{i+1/2})
-\dtdx (\widetilde F_{i+1/2} - \widetilde F_{i-1/2}),
\]
where $\W_{i-1/2} = Q_i^n - Q_{i-1}^n$ and the ``correction flux'' is
\[
\widetilde F_{i-1/2} = \half |\bar u| 
\left(1 - \dtdx |\bar u|\right) \widetilde \W_{i-1/2},
\]
with the limited waves $\widetilde \W$ defined by
\[
\widetilde \W_{i-1/2} = \phi(\theta_{i-1/2}) \W_{i-1/2}.
\]
The ratio $\theta_{i-1/2}$ is defined in (6.35) and the
function $\phi$ might be one of limiters from (6.39).

% uncomment the next two lines if you want to insert solution...
%\vskip 1cm
%{\bf Solution:}

% insert your solution here!

%--------------------------------------------------------------------------
\newpage

\vskip 1cm
\hrule
{\bf Programming problem}

Modify the IPython notebook \verb+$AM574/notebooks/Advection_Examples2.ipynb+
to create a new notebook \verb+Advection_Limiters.ipynb+ that illustrates
your solutions to the following:

\renewcommand{\theenumi}{\alph{enumi}}  % use letters for enumeration
\begin{enumerate}

\item Implement the wave limiter methods for advection, as described in 
the previous problem.   Note that it's impossible to use half-integer
indices, so you might want to declare arrays such as \verb+Ftilde+ in which 
\verb+Ftilde[i]+ holds the correction flux $\widetilde F_{i-1/2}$.  (This is
the convention used in Clawpack --- the index $i$ often refers to
information at the left edge of the cell $x_{i-1/2}$.)

Copy the \verb+upwind+ function definition to a new cell in the notebook and
modify it to create a new function \verb+wave_limiter+ that has one
additional argument \verb+phi+ in the calling sequence, so that a limiter
function $\phi(\theta)$ can be passed in.  The function $\phi$ might be one
of those listed in (6.39a,b) in the book.  

For example:
\begin{verbatim}
    def phi_minmod(theta):
        return(max(0., min(theta,1)))
\end{verbatim}
would define the minmax limiter, and then
\begin{verbatim}
    figs = wave_limiter(x,tfinal,nsteps,u,qtrue,nplot,phi_minmod)
\end{verbatim}
should solve the problem using the minmod wave-limiter method and return a
list of figures to animate.

\item Test your function by writing a notebook that produces 
animations up to time $t=10$ that
correspond to Figures 6.2 and 6.3 from the book.
This requires trying out several different limiter functions with 2 sets of
initial conditions.  

\end{enumerate} 

Submit your notebook \verb+Advection_Limiters.ipynb+.

{\bf Note:} Before submitting your notebook, clear all the output (From the
``Cell'' menu select ``All Output'' and then ``Clear''), and then save the
notebook.  Otherwise png figures are stored in the notebook file and it may
be very large.  

\vskip 15pt
{\bf Installing Clawpack.} You don't need to turn in anything for this, but
you should make sure you have Clawpack installed and working for future
assignments. 

I posted two videos at
\url{http://faculty.washington.edu/rjl/classes/am574w2017/lectures.html}
with tips on installing and getting started with running the code and
changing the inputs.

You might want to also clone the {\tt apps} repository, see
\url{http://www.clawpack.org/apps.html}.  The directory {\tt
apps/fvmbook/chap6} has the Clawpack code that creates Figures 6.2 and 6.3
(newly updated for Version 5.4.0).  You might want to use the plots you can
create in this directory for comparison.


\end{document}

