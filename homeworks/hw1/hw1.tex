\documentclass[11pt]{article}

\usepackage{graphicx}
\usepackage{amsmath,amsfonts,amssymb}

\usepackage{hyperref}  % for urls and hyperlinks


\setlength{\textwidth}{6.2in}
\setlength{\oddsidemargin}{0.3in}
\setlength{\evensidemargin}{0in}
\setlength{\textheight}{9in}
\setlength{\voffset}{-1in}
\setlength{\headsep}{26pt}
\setlength{\parindent}{0pt}
\setlength{\parskip}{5pt}

% input some useful macros from RJLmacros.tex:
\input{RJLmacros}

\begin{document}

\hfill\vbox{\hbox{AMath 574}\hbox{Homework 1}
\hbox{Due by 11:00pm on January 19, 2017}}

For submission instructions, see:

\url{http://faculty.washington.edu/rjl/classes/am574w2017/homework1.html}


%--------------------------------------------------------------------------
\vskip 1cm
\hrule
{\bf Problem \#2.7 in the book}


% uncomment the next two lines if you want to insert solution...
%\vskip 1cm
%{\bf Solution:}

% insert your solution here!


%--------------------------------------------------------------------------
\vskip 1cm
\hrule
{\bf Problem \#3.1 in the book} You might want to do Problem 3.2 first.


% uncomment the next two lines if you want to insert solution...
%\vskip 1cm
%{\bf Solution:}

% insert your solution here!


%--------------------------------------------------------------------------
\vskip 1cm
\hrule
{\bf Problem \#3.2 in the book}

You can use Matlab for this one, but I suggest you try writing the program 
in Python.  A Jupyter notebook will be provided to help you get started.

Note that the module {\tt numpy.linalg} contains an {\tt eig}
function similar to Matlab.

% uncomment the next two lines if you want to insert solution...
%\vskip 1cm
%{\bf Solution:}

% insert your solution here!


%--------------------------------------------------------------------------

\vskip 1cm
\hrule
{\bf Problem \#3.3 in the book}

Following the sort of thing done in 
script \verb+problem_3_5.py+ might be useful if you want to insert a
figure in your solution, or you can draw with another programming language, or
sketch the solution by hand and scan.


% uncomment the next two lines if you want to insert solution...
%\vskip 1cm
%{\bf Solution:}

% insert your solution here!

% to insert a figure named X.png, you might use this...
% \hfil\includegraphics[width=4.0in]{X.png}\hfil


%--------------------------------------------------------------------------

\vskip 1cm
\hrule
{\bf Problem \#3.5 in the book}

The script \verb+problem_3_5.py+ was used to generate this figure:

% you need to run
%     python problem_3_5.py
% at the command line to generate the figure inserted here:
\hfil\includegraphics[width=3.5in]{problem_3_5.png}\hfil

To solve this problem, determine the states $A,~ B, ~ \ldots,~ M$ and also
the times $t_1,~t_2,~t_3$.  The times can be written in terms of the
parameters $\rho_0$ and $K_0$, which were not stated in the problem.

For example,
\[
A = \bcm 0 \\ 0 \ecm, \quad, B = \bcm 1 \\ 0 \ecm, \quad, 
C = \bcm 0 \\ 0 \ecm, \quad \ldots
\]

% Note that bcm and ecm (begin and end centered matrices) are defined in 
% RJLmacros.tex

% uncomment the next two lines if you want to insert solution...
%\vskip 1cm
%{\bf Solution:}

% insert your solution here!


%--------------------------------------------------------------------------
\vskip 1cm
\hrule
{\bf Problem \#3.5A}
Solve \#3.5 with {\em periodic} boundary conditions instead of reflecting
walls.  Sketch the solution in the $x$--$t$ plane
up to at least time $t_3$ (as in \#3.5 the time
the right-going wave from $x_0=1$ hits the right boundary) 
and indicate the state in each section.  You might want to modify the
script \verb+problem_3_5.py+ to make the plot.


% uncomment the next two lines if you want to insert solution...
%\vskip 1cm
%{\bf Solution:}

% insert your solution here!



%--------------------------------------------------------------------------
\vskip 1cm
\hrule
{\bf Problem \#4.1 in the book}


% uncomment the next two lines if you want to insert solution...
%\vskip 1cm
%{\bf Solution:}

% insert your solution here!


%--------------------------------------------------------------------------
\vskip 1cm
\hrule
{\bf Problem \#4.2 in the book}


% uncomment the next two lines if you want to insert solution...
%\vskip 1cm
%{\bf Solution:}

% insert your solution here!

\end{document}

